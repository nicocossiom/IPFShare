\chapter*{Resumen}

% <<Aquí va el resumen del TFG. Extensión máxima 2 páginas.>>
%%--------------
% \newpage
%%--------------
IPFS, también conocido como Protocolo de Sistema de Archivos Interplanetario, 
es un protocolo de red y un sistema de archivos diseñado para hacer
la web más rápida, segura y abierta. Este sistema permite a los usuarios no 
solo recibir, sino también alojar contenido en una red P2P completamente descentralizada.

IPFS tiene varias ventajas clave. A diferencia de protocolos como HTTP, en IPFS los archivos se identifican 
por su contenido en lugar de por su ubicación. Esta característica permite a cualquier nodo de la red 
convertirse en proveedor de contenido dentro de ella, lo que se traduce en una mayor eficiencia, seguridad, 
escalabilidad y resiliencia para el almacenamiento y distribución de datos.
IPFS facilita la creación de aplicaciones descentralizadas (dApps) al proporcionar herramientas como
un sistema de almacenamiento de archivos distribuido y un sistema de nombres descentralizado (IPNS) para la web. Al mismo tiempo, promueve el desarrollo de aplicaciones resistentes a la censura y una web verdaderamente abierta y descentralizada.

Este trabajo de fin de grado se divide en dos partes:

La primera consiste en el estudio del ecosistema de IPFS. Se abarca desde su arquitectura, algoritmo de intercambio de bloques, 
identificación basada en contenido, hasta su estructura de datos. Se analizan ejemplos de casos de uso en la 
Web3, como la distribución descentralizada de contenido, el almacenamiento de datos en la cadena de bloques y
la publicación de datos permanentes.

La segunda parte del trabajo consiste en la creación de un sistema de intercambio de archivos basado en IPFS. 
Este sistema cuenta con características como la encriptación de archivos usando JWE (Encriptación Web JSON), 
verificación de autoría y firma de archivos usando JWS (Firma Web JSON). También se incorporan características 
como la gestión automática de claves y un sistema de registro de usuarios basado en bases de datos
descentralizadas, en particular en OrbitDB.

La aplicación desarrollada funciona en sistemas operativos Windows, MacOS y Linux. 
Mediante una interfaz de comandos de consola los usuarios pueden compartir archivos de manera segura y privada sin la necesidad de depender de servidores centralizados. 


\chapter*{Abstract}

% <<Abstract of the Final Degree Project. Maximum length: 2 pages.>>

%%%%%%%%%%%%%%%%%%%%%%%%%%%%%%%%%%%%%%%%%%%%%%%%%%%%%%%%%%%
%% Final del resumen. 
%%%%%%%%%%%%%%%%%%%%%%%%%%%%%%%%%%%%%%%%%%%%%%%%%%%%%%%%%%%

IPFS, also known as the InterPlanetary File System, is a network protocol and file system designed to make the web faster, more secure and open. This system allows users not only to receive but also to host content on a fully decentralized peer-to-peer network.

IPFS has several key advantages. Unlike protocols like HTTP, in IPFS, files are identified by their content rather than their location. This feature allows any node in the network to become a content provider, resulting in greater efficiency, security, scalability and resilience for data storage and distribution.

IPFS facilitates the creation of decentralized applications (dApps) by providing tools such as a distributed file storage system and a decentralized naming system (IPNS) for the web. At the same time it promotes the development of censorship-resistant applications as well as a truly open and decentralized web.

This undergraduate thesis is divided into two parts:

The first part consists of the study of the IPFS ecosystem. From its architecture, block exchange algorithm, content-based addressing, to its data structure. Examples of use cases in Web3, such as decentralized content distribution, blockchain-based data storage, and permanent data publishing, are also analyzed.

The second part of the thesis involves the creation of a file-sharing system based on IPFS. This system features file encryption using JSON Web Encryption (JWE), authorship verification enabled by file signing using JSON Web Signatures (JWS). It also incorporates features such as automatic key management and a user registration system based on decentralized databases, particularly OrbitDB.

The developed application supports Windows, MacOS, and Linux operating systems. Through a command-line interface, users can securely and privately share files without relying on centralized servers.