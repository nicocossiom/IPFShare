\chapter{IPFS}
\label{chap:ipfs}
\section{¿Qué es IPFS?}
\subsection{Introducción}
% Briefly introduce the concept of IPFS and its significance in the modern era of the internet.
% Provide a background on why IPFS was created and its aims.
\subsection{Fundamentos}
% Describe the key concepts and terminologies involved in IPFS, such as Content-addressed, Peer-to-peer (P2P) network, and Merkle DAG.
% Explain how these concepts work together to make IPFS a decentralized and distributed file system.
\subsection{Arquitectura}
% Describe the core components of IPFS architecture, such as IPFS node, IPFS client, IPFS gateway, and IPFS cluster.
% Explain the role of each component in the IPFS network.
\subsection{Modelo de datos}
% Explain the IPFS data model, which involves content addressing and hash-based naming.
% Discuss the benefits of the IPFS data model, such as data integrity, data verifiability, and content-based addressing.
\subsection{Distribución de contenido}
% Describe how IPFS handles content distribution and how it differs from the traditional client-server model.
% Explain how IPFS content distribution works in a P2P network.

\section{Ecosistema en torno a IPFS}

\subsection{Introducción}
% Introduce the concept of the IPFS ecosystem and why it is important.
% Briefly discuss the role of the ecosystem in the development and adoption of IPFS.
\subsection{Proyectos basados en IPFS} % Hablar de las DAPPS
% Provide an overview of the various IPFS projects that exist.
% Categorize the projects into different areas of application, such as storage, web hosting, content distribution, and decentralized applications (dApps).
% Provide examples of notable projects in each category and briefly describe their goals and features.
\subsection{Herramientas y librerías de IPFS}
% Discuss the various tools and libraries that are available to developers working with IPFS.
% Categorize the tools and libraries into different areas, such as API clients, command-line interfaces, development frameworks, and libraries for integrating IPFS into other applications.
% Provide examples of notable tools and libraries in each category and briefly describe their features and capabilities.
\subsection{Comunidades en torno a IPFS}
% Discuss the various communities that exist around IPFS.
% Categorize the communities into different areas, such as developer communities, user communities, and governance communities.
% Provide examples of notable communities in each category and briefly describe their goals and activities.
\subsection{Integraciones de IPFS}
% Discuss the various ways in which IPFS is being integrated with other technologies and platforms.
% Provide examples of notable integrations, such as with Ethereum, Filecoin, and the InterPlanetary Naming System (IPNS).
% Discuss the potential benefits of these integrations and their implications for the broader ecosystem.
