\chapter{Introducción}
%%---------------------------------------------------------
El presente Trabajo de Fin de Grado (TFG) se centra en el desarrollo de un sistema de intercambio de ficheros basado en
IPFS (InterPlanetary File System)\cite{IPFSPowersDistributed}. A continuación, se describen las motivaciones y necesidades que han llevado a la realización de este proyecto.
\section{Motivación y necesidad}

El desarrollo de un sistema de intercambio de ficheros basado en IPFS se encuentra en la confluencia de varias tendencias
tecnológicas y sociales que están dando forma al futuro de la web. En particular, este proyecto se relaciona estrechamente
con el avance hacia la \textit{Web3}\cite{Web32023}, una visión de una Internet más descentralizada, segura y resistente a la censura.
En esta sección, exploraremos cómo un sistema de intercambio de archivos encaja en este nuevo panorama y por qué es relevante para el progreso de la Web3.

Los servicios de almacenamiento y compartición de archivos actuales, como Google Drive, Dropbox, Microsoft OneDrive y otros proveedores de almacenamiento en la
nube, son centralizados y, aunque populares y ampliamente utilizados debido a su facilidad de uso, accesibilidad y confiabilidad,
presentan ciertos problemas y limitaciones. Los usuarios dependen de una sola entidad para almacenar y gestionar sus archivos, lo que puede
generar problemas si la empresa experimenta fallos técnicos, cambia sus políticas de uso, o se convierte en el objetivo de un ataque.
Además, esto otorga a estas empresas un gran poder sobre los datos de los usuarios, lo que puede conducir a problemas de privacidad y control de la información.

La alternativa a estos servicios centralizados es el uso de tecnologías \textit{peer-to-peer} (p2p). Existen varias tecnologías p2p que permiten compartir archivos
entre usuarios sin necesidad de un proveedor central como los previamente mencionados, el más famoso y conocido siendo BitTorrent. Sin embargo, estas tecnologías no son adecuadas para el intercambio de archivos entre usuarios no conocidos, ya que requieren que los usuarios confien en que los archivos que se comparten son los que se anuncian.

Esto es algo que resuelve el Inter Planetary File System (IPFS). Es un sistema de archivos distribuido tipo peer-to-peer que conecta dispositivos informáticos en un único sistema de archivos. Funciona como una mezcla de BitTorrent y Git, proporcionando almacenamiento de bloques y enlaces direccionados por contenido en una estructura llamada DAG de Merkle. IPFS permite construir sistemas de archivos versionados, blockchains y una Web permanente. Combina una tabla hash distribuida, intercambio de bloques incentivado y espacio de nombres auto-certificante, sin puntos únicos de falla ni necesidad de confianza entre nodos \cite{benetIPFSContentAddressed2014}.

En este proyecto se usará IPFS como bloque central, íntrinseco y necesario sobre el que construirá el sistema. Parte del diseño estará basado
en el sistema ideado por Hsiao-Shan Huang, Tian-Sheuan Changen y Jhih-Yi Wu, descrito en \textit{A Secure File Sharing System Based on IPFS and Blockchain} \cite{huangSecureFileSharing2020}, aunque se realizarán modificaciones para adaptarlo a las necesidades y visión del proyecto. Esto se especificará en la sección \ref{sec:arquitectura_del_sistema} \textit{Arquitectura del sistema}.

\section{Objetivos y alcance}
El objetivo principal de este proyecto es el desarrollo de un sistema de intercambio de ficheros basado en IPFS, mediante el desarrollo una aplicación de escritorio. Este sistema debe permitir a los usuarios compartir archivos de forma segura y confiable, sin necesidad de un proveedor central.
Para ello se deben cumplir los siguientes objetivos:
\begin{itemize}

      \item Investigar sobre IPFS y su funcionamiento para entender cómo funciona el protocolo
            y cómo se puede utilizar para el sistema propuesto.
      \item Investigar sobre el ecosistema en torno a IPFS, con objetivo de comprender
            la madurez y viabilidad de esta tecnología, así como de las herramientas basadas en esta
            que se pueden utilizar para el sistema propuesto.
      \item Diseñar una arquitectura para el sistema de intercambio en torno a las tecnologías y herramientas seleccionadas.
      \item Implementación de un prototipo funcional en una o varias plataformas.
      \item Implementar la encriptación y control de acceso a los archivos para garantizar la
            seguridad de los mismos.
      \item Desarrollo de integración de con la red global de IPFS para
            aumentar la consistencia de los datos compartidos.
      \item Analizar la viabilidad de IPFS en base a la experiencia obtenida en el desarrollo
            del prototipo.
      \item Analizar posibles mejoras y ampliaciones del sistema propuesto.

\end{itemize}

Por tanto pese a que el objetivo principal es el desarrollo de un sistema de intercambio de ficheros basado en IPFS,
también se realizará una labor invesitigativa sobre la tecnología IPFS y su ecosistema, con el objetivo de comprender esta tecnología
y su viabilidad como alternativa al protocolo HTTP en la que se basa internet en la actualidad.
